%%%%%%%%%%%%%%%%%%%%%%%%%%%%%%%%%%%%%%%%%%%%%%%%%%%%%%%%%%%%%%%%%%%%%%%%%%%%%%%%%%%%%%%%%%%%%
%%									ANNEXES 												%
%%%%%%%%%%%%%%%%%%%%%%%%%%%%%%%%%%%%%%%%%%%%%%%%%%%%%%%%%%%%%%%%%%%%%%%%%%%%%%%%%%%%%%%%%%%%%

%%%%%%%%%%%%%%%%%%%%%%%%%%%%%%%%%%%%%%%%%%%%%%%%%%%%%%%%%%%%%%%%%%%%%%%%%%%%%%%%%%%%%%%%%%%%%


\chapter{Charge Collection Efficiency in Irradiated Silicon Pads}
\label{sec:CCEirr}

In this Section estimates of the expected charge collection efficiencies 
for irradiated pads will be derived under some simplistic assumptions.

\section{Introduction}
In this part the assumptions made will be outlined after having reminded how signal is formed when 
carrier moves towards the collecting electrodes.
\subsection{Ramo theorem}
\label{sec:CCEramo}
The instantaneous current $i(t)$ appearing on the electrodes of a silicon pad can be expressed 
in terms of the charge of the carriers $q_{e,h}$, the drift velocity $\vec{v}_{e,h}$  and the 
weighting field $\vec{E}_w$. For the sake of simplicity the time/position/temperature/voltage 
dependence of the drift velocity are here omitted, as well the dependence on position of the 
weighting field.
The following formula holds separately for electrons and holes:

\begin{equation}
i(t) = q\,\vec{v}\cdot\vec{E}_w
\end{equation} 

\subsection{Assumptions to simplify the calculation}
\label{sec:assumptions}
A simple 1D diode will be considered; its bulk depth is equal to $w$. The direction of the carriers 
drift will be identified with $z$; electrons will move toward the $z=0$ position, the holes toward $z=w$.
With the above assumptions, when focusing on the electrode collecting electrons 
the weighting field is simply equal to:

\begin{equation}
\vec{E}_w=\dfrac{1}{w}\hat{z}
\end{equation}

Diffusion is neglected as well as temperature dependence for whatsoever variable. 
The drift velocity is assumed to be saturated: $v_{e,h}=v^{(sat)}_{e,h}$; still the values can be different for electrons and holes. 
With the above assumptions the vectorial drift velocity is simply equal to:

\begin{equation}
\vec{v}_{e,h}=(\mp)v^{(sat)}_{e,h}\hat{z}
\end{equation}

The trapping effect will be modeled through an exponential attenuation with time of drifting carriers:

\begin{equation}
q_{e,h}(t) = q_{e,h}(0)e^{-\dfrac{t}{\tau_{e,h}}}
\end{equation}

The trapping time $\tau_{e,h}$ is related to the fluence $\phi$ through:

\begin{equation}
(\tau_{e,h})^{-1}=\beta_{e,h}\phi
\end{equation}
$\beta_{e,h}$ are the trapping constants.

The event of trapping and de-trapping within the current integration time will be neglected.

The passage of a MIP through the entire sensor thickness will be considered. The 
rate of charge created per unit length is $\mathcal{Q} (\sim \dfrac{80e}{\mu m})$. The total charge 
released in the silicon bulk by the MIP is $Q_0=\mathcal{Q}\,w$

\section{From instantaneous current to charge on electrodes}
\label{sec:charge}
Under the assumptions made in Section~\ref{sec:assumptions} the instantaneous current $i(t)$ 
from electrons and holes is simply equal to:

\begin{equation}
i_{e,h}(t)=e\,\dfrac{v^{(sat)}_{e,h}}{w}e^{-\dfrac{t}{\tau_{e,h}}}
\label{eq:final_current}
\end{equation}

To get the charge on electrodes the Equation~\ref{eq:final_current} has to be integrated over the collection time $t_{coll}$ and over all the possible initial $z$ position of the carriers.
Given that the drift velocities are constant the collection times are equal to:

\begin{equation}
t_{coll_{e,h}}=\dfrac{(z,w-z)}{v^{(sat)}_{e,h}}
\end{equation}

The charge appearing on the electrode due to electrons  is $Q_e$:

\begin{equation}
Q_e = \int_0^wd{\rm z}\int_0^{\dfrac{z}{v^{(sat)}_{e}}}{\rm d}t\mathcal{Q}\dfrac{v^{(sat)}_{e}}{w}e^{-\dfrac{t}{\tau_{e}}}
\label{eq:Qe_to_integrate}
\end{equation}
Integrating Equation~\ref{eq:Qe_to_integrate}, and introducing the collecting distance $d_e=v^{(sat)}_{e}\,\tau_{e}=\dfrac{v^{(sat)}_{e}}{\beta_{e,h}\phi}$, the charge  $Q_e$ due to electrons is found to be:

\begin{equation}
Q_e=Q_0\dfrac{d_e}{w}\Big[1-\dfrac{d_e}{w}\Big(1-e^{-\dfrac{w}{d_e}}\Big)\Big]
\label{eq:Qe}
\end{equation}

Similarly, the charge appearing on the electrode due to holes is $Q_h$:
\begin{equation}
Q_h=Q_0\dfrac{d_h}{w}\Big[1-\dfrac{d_h}{w}\Big(1-e^{-\dfrac{w}{d_h}}\Big)\Big]
\label{eq:Qh}
\end{equation}

The charge collection efficiency, hence, is  $CCE=\dfrac{Q}{Q_0}$:
\begin{equation}
CCE=\dfrac{Q}{Q_0}=\Big[\dfrac{d_e+d_h}{w}\Big]-\Big(\dfrac{d_e}{w}\Big)^2\Big(1-e^{-\dfrac{w}{d_e}}\Big)-\Big(\dfrac{d_h}{w}\Big)^2\Big(1-e^{-\dfrac{w}{d_h}}\Big)
\label{eq:CCE}
\end{equation}

From Equation~\ref{eq:Qe} and~\ref{eq:Qh} it is also possible to get the contribution to $CCE$ 
for electrons and holes separately. 

\section{Predictions for interesting cases: unirradiated vs large fluences}

\subsubsection{Unirradiated sensor}

In the case of unirradiated sensors the fluence $\phi$ is zero, so the trapping time $\tau$ and the 
collecting distance $d$ are infinite. Working out the limits the $CCE$ is found to be:
\begin{equation}
\begin{split}
\lim_{\phi \to 0}CCE & \sim \\ \lim_{d_e,d_h\to \infty}   \Big\{\Big[\dfrac{d_e+d_h}{w}\Big]&-\Big(\dfrac{d_e}{w}\Big)^2\Big(\dfrac{w}{d_e}-\dfrac{1}{2}\big(\dfrac{w}{d_e}\big)^2+\dfrac{1}{6}\big(\dfrac{w}{d_e}\big)^3\Big)-\Big(\dfrac{d_h}{w}\Big)^2\Big(\dfrac{w}{d_h}-\dfrac{1}{2}\big(\dfrac{w}{d_h}\big)^2+\dfrac{1}{6}\big(\dfrac{w}{d_h}\big)^3\Big)\Big\}=\\
&\dfrac{1}{2}+\dfrac{1}{2}1-\dfrac{w}{6}\Big(\dfrac{1}{d_e}+\dfrac{1}{d_h}\Big)=1-\dfrac{w}{6}\Big(\dfrac{1}{d_e}+\dfrac{1}{d_h}\Big)
\end{split}
\label{eq:CCE_unirr}
\end{equation}
To summarise: $CCE(\phi\to 0)\sim1-\dfrac{w}{6}\Big(\dfrac{1}{d_e}+\dfrac{1}{d_h}\Big)$; for infinite 
collecting distances the $CCE$ is exactly one. 
It is interesting to notice that there's a contribution 1/2 from electrons and 1/2 from holes to the 
$CCE$. Neglecting holes in case of no  trapping for electrons the $CCE$  would be merely 1/2: 
all the holes would be trapped and screen on average half of the total charge $Q_0$.

\subsubsection{Large fluences}
In the case of sensors irradiated to large fluences  the 
collecting distance $d$ is negligible with respect to the sensor thickness $w$.

Working out the limits the $CCE$ is found to be:
\begin{equation}
\begin{split}
\lim_{\phi \to \infty}CCE & \sim \\ 
\lim_{d_e,d_h\to 0}  \Big\{\Big[\dfrac{d_e+d_h}{w}\Big]& - \Big(\dfrac{d_e}{w}\Big)^2 -\Big(\dfrac{d_h}{w}\Big)^2\Big\}\\
\sim & \Big[\dfrac{d_e+d_h}{w}\Big]
\end{split}
\label{eq:CCE_irr}
\end{equation}

To summarise: $CCE(\phi\to \infty)\sim\Big[\dfrac{d_e+d_h}{w}\Big]$; for zero 
collecting distances the $CCE$ is exactly zero. 
The result  in Equation~\ref{eq:CCE_irr} means that in case of heavily irradiation the charge 
can be effectively collected only within a distance from the electrode that is equal to the collection distance $d$; all the remaining $w-d$ have a negligible contribution. As for the unirradiated case, if holes are neglected the $CCE$ would be only about 1/2 of what it should be (in general
 the collection distances are different for electrons and holes).

\section{Estimates for some scenarios}

In Table~\ref{tab:cce_scenarios} some estimates for the CCE at certain fluences $\Phi$ for 
certain diode thicknesses $w$ are reported.

\begin{table}[!htbp]
\centering
\begin{tabular}{ccc}
\hline
$w$ [$\mu$m] & $\Phi$ [1$\times10^{15}$~n$_{\rm eq}$/cm$^2$] & CCE [\%] \\
\hline
\hline
50 & 3 & 75 \\
50 & 5 & 	63 \\
50 & 7 & 54 \\
50 & 	10 & 	45 \\
50 & 15 & 34 \\
50 & 20 & 27\\
\hline
100 & 1 & 82 \\
100 & 3 & 59 \\
100 & 5 & 	45 \\
100 & 7 & 36 \\
100 & 10 & 27 \\
100 & 20 & 15 \\
\hline
200 & 0.1 & 96\\
200 & 0.2 & 92\\
200 & 0.5 & 82\\
200 & 1 & 69 \\
200 & 3 & 40 \\
200 & 5 & 	27 \\
200 & 7 & 21 \\
200 & 10 & 15 \\
\hline
\end{tabular}
\caption{\label{tab:cce_scenarios}CCE estimates at fluences $\Phi$ for 
certain diode thicknesses $w$.}
\end{table}

It can be seen that already for moderate fluences, 1-2$\times10^{14}$~n$_{\rm eq}$/cm$^2$, 
the signal amplitude loss is sizeable if the detector is 200~$\mu$m thick. 
For fluences in the range of 5-10$\times10^{15}$~n$_{\rm eq}$/cm$^2$ only very thin diodes 
can still deliver more than 1/4 of the original signal amplitude. 

It is interesting to observe that for 1-2$\times10^{16}$~n$_{\rm eq}$/cm$^2$ there's little or no 
difference in expected CCE between 100 and 50 $\mu$m thick diodes, since charge is collected 
within a very thin layer close to the collected electron, the thickness of this layer being smaller 
than the diode one $w$.

\chapter{Trap occupation probability}
\label{sec:trapoccprob}

The purpose of this section is to motivate the variations in the electric field profiles in Sec.~\ref{sec:Efieldmodelcomparisons} with changes in the defining parameters of the two-trap radiation damage model presented in Sec.~\ref{sec:Efieldmodelcomparisons}.  Many of the concepts described here are discussed in more detail in Ref.~\cite{LUTZ1996}.

Variations in the electric field profile are driven by modifications to the space charge distribution.   The contribution to the space charge density from radiation damage is due to charged traps.  The Chiochia model has two non-degenerate traps with two possible charge states, one of them neutral.  A trap is occupied if it can emit an electron; a donor-like trap is charged if it is not occupied and an acceptor-like trap is charged if it is occupied.  For one of the two traps $t$ in the Chiochia model, the occupation probability $P_t$ can be estimated from its energy level $E_t$ and its electron and hole capture cross sections $\sigma_{n,p}$. Once the trap occupation probability is known, its average charge state $Q_t$  can be calculated.  In particular, $Q_t = (1-P_t)$ for donors and $Q_t=-P_t$ for acceptors.

%Only non-degenerate traps with 2 possible charge states, one of them neutral, are treated.  A trap is called ``donor'' if it can be either neutral or positively charged; on the contrary, a trap is called ``acceptor'' if it can be either neutral or negatively charged. In the following we will refer to a trap in the non neutral state as ``ionized''.

It is useful to start with the limiting cases.  When $E_t\sim E_c$ a donor trap is a shallow donor, which is most of the time ionized ($Q_t\approx +1$); on the contrary, when $E_t\sim E_v$ an acceptor trap is a shallow acceptor, which is most of the time ionized ($Q_t\sim -1$).  In contrast, the occupation probability is exactly 50\% at the Fermi energy level.  The intrinsic energy level $E_i= \dfrac{E_v+E_c}{2}+\dfrac{1}{2}k_BT\ln\Big(\dfrac{N_v}{N_c}\Big)\sim$~0.534~eV is almost exactly half way between the valence and conduction bands: $E_g=E_c-E_v$=1.09~eV.  In the Chiochia model, the acceptor (a) and donor (d) states are very close to the intrinsic energy: $E_a=0.525$~e and  $E_d=0.48$~eV ($kT\sim 0.023$~eV).  As the acceptor trap is only 0.031~eV above the intrinsic level (about 1.4~$kT$) and the donor trap is only 0.054~eV below the  intrinsic level (about 2.4~$kT$), small changes to the model parameters can result in significant changes in the space charge.

%If the traps are within few $kT$ from $E_i$ the trap occupation probability $P_t$ will be not simply 0 or 1. For the sake of radiation damage we are indeed interested in traps close to the intrinsic energy level, $E_i$. For reference the TCAD simulations presented in this paper where run at $T=263.15$~K; hence the value of thermal energy  was $V_{th}\sim 0.023$~eV, to be compared to the intrinsic energy level, $E_i= \dfrac{E_v+E_c}{2}+\dfrac{1}{2}k_BT\ln\Big(\dfrac{N_v}{N_c}\Big)\sim$~0.534~eV; for the sake of completeness: $E_g=E_c-E_v$=1.09~eV, $N_c$=2.3$\times$10$^{19}/\text{cm}^3$ and $N_v$=8.54$\times$10$^{18}/\text{cm}^3$. For example, in Chiochia model~\cite{Chiochia1}  we have $E_a=0.525$~eV and  $E_d=0.48$~eV. So the acceptor trap is 0.031~eV above the intrinsic level (about 1.4~$V_{th}$); the donor trap is 0.054~eV below the  intrinsic level (about 2.4~$V_{th}$).

The full expression for the occupation probability for a trap $t$ is given by~\cite{LUTZ1996}

\begin{equation}
  P_t=\dfrac{1}{1+\dfrac{c_pp+c_nn_ix_t}{c_nn+c_pn_i/x_t}},
  \label{eq:Pt}
\end{equation}

%\begin{equation}
 % (1-P_t)=\dfrac{1}{1+\dfrac{c_nn+c_pn_i/x_t}{c_pp+c_nn_ix_t}}
 % \label{eq:oneminusPt}
%\end{equation}

\noindent where $x_t(E_t)=e^{\dfrac{E_t-E_i}{kT}}$, $c_{n,p}=v_{th_{n,p}}\sigma_{n,p}$ for electron/hole thermal velocities $v_{th_{n,p}}$, and $n_i$ denotes the intrinsic carrier concentration.  If the bulk is depleted, hence the electron and hole concentrations are negligible with respect to the intrinsic one ($n,p\ll n_i$), then the Eq.~\ref{eq:Pt} simplifies to:

\begin{equation}
P_t=\dfrac{c_p/x_t}{c_nx_t+c_p/x_t}=\dfrac{1}{1+\dfrac{c_nx_t}{c_p/x_t}}
\label{eq:simplePt}
\end{equation}

%With ~\cite{LUTZ1996} we define:
%\begin{equation}
 % x(E)=e^{\dfrac{E-E_i}{kT}}
  %\label{eq:xE}
%\end{equation}

%and the capture coefficients for electrons and holes, $c_{n,p}=v_{th_{n,p}}\sigma_{n,p}$ ($v_{th_{n,p}}$ are the elecontron/hole thermal velocities).

%Using the above definitions, we can then write $P_t$ and $1-P_t$:

%\begin{equation}
 % P_t=\dfrac{1}{1+\dfrac{c_pp+c_nn_ix_t}{c_nn+c_pn_i/x_t}}
  %\label{eq:Pt}
%\end{equation}

%\begin{equation}
 % (1-P_t)=\dfrac{1}{1+\dfrac{c_nn+c_pn_i/x_t}{c_pp+c_nn_ix_t}}
 % \label{eq:oneminusPt}
%\end{equation}

%($n_i$ is the intrinsic concentration).

%If the bulk is depleted, hence the electron and hole concentrations are negligible with respect to the intrinsic one ($n,p<<n_i$), then the Equations~\ref{eq:Pt} and~\ref{eq:oneminusPt} simplify to:

%\begin{equation}
%P_t=\dfrac{c_p/x_t}{c_nx_t+c_p/x_t}=\dfrac{1}{1+\dfrac{c_nx_t}{c_p/x_t}}
%\label{eq:simplePt}
%\end{equation}

%and:
%\begin{equation}
%1-P_t=\dfrac{c_nx_t}{c_nx_t+c_p/x_t}=\dfrac{1}{1+\dfrac{c_p/x_t}{c_nx_t}}
%\label{eq:simpleoneminusPt}
%\end{equation}

The space charge density (in units of elementary charge per volume) is then given by $\mathcal{Q}_t = N_t(1-P_t)$ for donors and $\mathcal{Q}_t = N_tP_t$ for acceptors.  By substituting Eq.~\ref{eq:simplePt} into these formulae, it is possible to assess the impact of changes in $E_a,E_d,\sigma_{d,n},\sigma_{d,p},\sigma_{a,n},\sigma_{a,p},\eta_a$, and $\eta_d$ on the space charge density.  The relationships are concisely summarized in Table~\ref{tab:trapEnergy} for the energy levels and Eq.~\ref{tab:cross_sections} for the capture cross-sections and introduction rates.  An upward or downward pointing arrow indicates how the parameter in the second columns of the table is changed.  Then, the penultimate and final columns indicate the trend in the occupation probability and average charge.  For $\eta$, a higher rate (lower) rate results in a higher (lower) space charge density: positive for donors, negative for acceptors.  The more negative the space charge, the stronger the electric field is near the front (bias electrode) side of the sensor and vice versa.

%Trap energy is defined in Silvaco\footnote{www.silvaco.com} as shown in Figure~\ref{fig:SilvacoTraps}. For consistency with what indicated in notes etc., we rename the trap energies in the following way: $E_a=E_{tA}$ and $E_d=E_{tD}$.

%\begin{figure}[htbp]
  %   \centering
    %    \includegraphics{figures/efield_maps/SilvacoTraps.png} 
     %   \caption{Definition of trap energy in Silvaco. \url{www.silvaco.com}}
      %        \label{fig:SilvacoTraps}
       %     \end{figure}

%Before moving to the detailed study of the impact of each parameter on trap occupation probability, hence space charge, it is worth comment quickly on the energy of the deep level. Let's start with donor traps. Moving its energy $E_t$ from the valence to the conduction band the occupation probability goes from 1 to 0, hence the status from neutral to positively charged. For the acceptor trap the opposite goes: moving the energy  $E_t$ from the valence to the conduction band the occupation probability goes from 1 to 0, hence the status from negatively  charged to neutral.

%So if we now use the $E_d$ and $E_a$ symbols we can summarise the situation as in Table~\ref{tab:trapEnergy}.


\begin{table}[htbp]
   \centering
   \begin{tabular}{lccccc} 
      Parameter   & Increasing? Decreasing? & closer to?& $P_t$ & $Q_t$  \\
      \hline
      \hline
      $E_d$   & $\nearrow$  &      $E_c$& $1\to0$ & $0\to+$  \\
      $E_d$    & $\searrow$ &      $E_v$     & $0\to1$ & $+\to0$  \\
      \hline
      $E_a$   & $\nearrow$ &      $E_v$& $0\to1$ & $0\to-$  \\
      $E_a$    & $\searrow$ &     $E_c$ & $1\to0$ & $-\to0$  \\
      \hline
   \end{tabular}
   \caption{Change in occupation probability $P_t$ and charge state $Q_t$ for changes of
   trap energy values.}
   \label{tab:trapEnergy}
\end{table}

%As already mentioned, all the changes mentioned above happen within few $kT$ around the intrinsic energy $E_i$.

%Let's now consider the introduction rate $\eta$. Looking at Eq.~\ref{eq:Qdonors} and~\ref{eq:Qacceptors} it is clear that the higher (lower) $\eta$ the higher (lower) the space charge density, positive for donors, negative for acceptors.

%Let's now focus on capture cross sections. The Table~\ref{tab:cross_sections} summarises the findings, based on Equations~\ref{eq:simplePt},~\ref{eq:simpleoneminusPt},~\ref{eq:Qdonors} and~\ref{eq:Qacceptors}.

\begin{table}[!htbp]
   \centering
   \begin{tabular}{lccccc} % Column formatting, @{} suppresses leading/trailing space
    Trap type&  Parameter   & Increasing? Decreasing? & $P_t$ & $Q_t$  \\
      \hline
      \hline
      \hline
     Donor & $\sigma_n$   & $\nearrow$  &       $1\to0$ & $0\to+$  \\
     Donor &$\sigma_n$    & $\searrow$ &      $0\to1$ & $+\to0$  \\
      \hline
     Donor & $\sigma_p$   & $\nearrow$  &       $0\to1$ & $+\to0$  \\
     Donor &$\sigma_p$    & $\searrow$ &      $1\to0$ & $0\to+$  \\
      \hline
      \hline
    Acceptor  & $\sigma_n$   & $\nearrow$ &       $1\to0$ & $-\to0$  \\
    Acceptor  & $\sigma_n$    & $\searrow$ &      $0\to1$ & $0\to-$  \\
      \hline
       Acceptor  & $\sigma_p$   & $\nearrow$ &       $0\to1$ & $0\to-$  \\
    Acceptor  & $\sigma_p$    & $\searrow$ &      $1\to0$ & $-\to0$  \\
      \hline
   \end{tabular}
   \caption{Change in occupation probability $P_t$ and charge state $Q_t$ for changes of
   trap cross sections values.}
   \label{tab:cross_sections}
\end{table}




%\chapter{Additional Model Predictions}
%\label{sec:additional}

%\chapter{Flavour Tagging Algorithm}
	%\blindtext
	%On rappelle que \gls{alpha} et \gls{gamma} sont liés par la relation~\eqref{eq:alphagamma}. Pour plus de détails, voir page~\pageref{eq:alphagamma}.


	%\blindtext
