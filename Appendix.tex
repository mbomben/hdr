%%%%%%%%%%%%%%%%%%%%%%%%%%%%%%%%%%%%%%%%%%%%%%%%%%%%%%%%%%%%%%%%%%%%%%%%%%%%%%%%%%%%%%%%%%%%%
%%									ANNEXES 												%
%%%%%%%%%%%%%%%%%%%%%%%%%%%%%%%%%%%%%%%%%%%%%%%%%%%%%%%%%%%%%%%%%%%%%%%%%%%%%%%%%%%%%%%%%%%%%

%%%%%%%%%%%%%%%%%%%%%%%%%%%%%%%%%%%%%%%%%%%%%%%%%%%%%%%%%%%%%%%%%%%%%%%%%%%%%%%%%%%%%%%%%%%%%


\chapter{Charge Collection Efficiency in Irradiated Silicon Pads}
\label{sec:CCEirr}

In this Section estimates of the expected charge collection efficiencies 
for irradiated pads will be derived under some simplicistic assumptions.

\subsection{Ramo theorem}
\label{sec:CCEramo}
The instantaneous current $i(t)$ appearing on the electrodes of a silicon pad can be expressed 
in terms of the charge of the carriers $q_{e,h}$, the drift velocity $\vec{v}_{e,h}$  and the 
weighting field $\vec{E}_w$. For the sake of simplicity the time/position/temperature/voltage 
dependence of the drift velocity are here omitted, as well the dependence on position of the 
weighting field.
The following formula holds separately for electrons and holes:

\begin{equation}
i(t) = q\,\vec{v}\cdot\vec{E}_w
\end{equation} 

\subsection{Assumptions to simplify the calculation}
\label{sec:assumptions}
A simple 1D diode will be considered; its bulk depth is equal to $w$. The direction of the carriers 
drift will be identified with $z$; electrons will move toward the $z=0$ position, the holes toward $z=w$.
With the above assumptions, when focusing on the electrode collecting electrons 
the weighting field is simply equal to:

\begin{equation}
\vec{E}_w=\dfrac{1}{w}\hat{z}
\end{equation}

Diffusion is neglected as well as temperature dependence for whatsoever variable. 
The drift velocity is assumed to be saturated: $v_{e,h}=v^{(sat)}_{e,h}$; still the values can be different for electrons and holes. 
With the above assumptions the vectorial drift velocity is simply equal to:

\begin{equation}
\vec{v}_{e,h}=(\mp)v^{(sat)}_{e,h}\hat{z}
\end{equation}

The trapping effect will be modeled through an exponential attenuation with time of drifting carriers:

\begin{equation}
q_{e,h}(t) = q_{e,h}(0)e^{-\dfrac{t}{\tau_{e,h}}}
\end{equation}

The trapping time $\tau_{e,h}$ is related to the fluence $\phi$ through:

\begin{equation}
(\tau_{e,h})^{-1}=\beta_{e,h}\phi
\end{equation}
$\beta_{e,h}$ are the trapping constants.

The event of trapping and de-trapping within the current integration time will be neglected.

The passage of a MIP through the entire sensor thickness will be considered. The 
rate of charge created per unit length is $\mathcal{Q} (\sim \dfrac{80e}{\mu m})$. The total charge 
released in the silicon bulk by the MIP is $Q_0=\mathcal{Q}\,w$

\section{From instantaneous current to charge on electrodes}
\label{sec:charge}
Under the assumptions made in Section~\ref{sec:assumptions} the instantaneous current $i(t)$ 
from electrons and holes is simply equal to:

\begin{equation}
i_{e,h}(t)=e\,\dfrac{v^{(sat)}_{e,h}}{w}e^{-\dfrac{t}{\tau_{e,h}}}
\label{eq:final_current}
\end{equation}

To get the charge on electrodes the Equation~\ref{eq:final_current} has to be integrated over the collection time $t_{coll}$ and over all the possible initial $z$ position of the carriers.
Given that the drift velocities are constant the collection times are equal to:

\begin{equation}
t_{coll_{e,h}}=\dfrac{(z,w-z)}{v^{(sat)}_{e,h}}
\end{equation}

The charge appearing on the electrode due to electrons  is $Q_e$:

\begin{equation}
Q_e = \int_0^wd{\rm z}\int_0^{\dfrac{z}{v^{(sat)}_{e}}}{\rm d}t\mathcal{Q}\dfrac{v^{(sat)}_{e}}{w}e^{-\dfrac{t}{\tau_{e}}}
\label{eq:Qe_to_integrate}
\end{equation}
Integrating Equation~\ref{eq:Qe_to_integrate}, and introducing the collecting distance $d_e=v^{(sat)}_{e}\,\tau_{e}=\dfrac{v^{(sat)}_{e}}{\beta_{e,h}\phi}$, the charge  $Q_e$ due to electrons is found to be:

\begin{equation}
Q_e=Q_0\dfrac{d_e}{w}\Big[1-\dfrac{d_e}{w}\Big(1-e^{-\dfrac{w}{d_e}}\Big)\Big]
\label{eq:Qe}
\end{equation}

Similarly, the charge appearing on the electrode due to holes is $Q_h$:
\begin{equation}
Q_h=Q_0\dfrac{d_h}{w}\Big[1-\dfrac{d_h}{w}\Big(1-e^{-\dfrac{w}{d_h}}\Big)\Big]
\label{eq:Qh}
\end{equation}

The charge collection efficiency, hence, is  $CCE=\dfrac{Q}{Q_0}$:
\begin{equation}
CCE=\dfrac{Q}{Q_0}=\Big[\dfrac{d_e+d_h}{w}\Big]-\Big(\dfrac{d_e}{w}\Big)^2\Big(1-e^{-\dfrac{w}{d_e}}\Big)-\Big(\dfrac{d_h}{w}\Big)^2\Big(1-e^{-\dfrac{w}{d_h}}\Big)
\label{eq:CCE}
\end{equation}

From Equation~\ref{eq:Qe} and~\ref{eq:Qh} it is also possible to get the contribution to $CCE$ 
for electrons and holes separately. 

\subsection{Predictions for interesting cases: unirradiated vs large fluences}

\subsubsection{Unirradiated sensor}

In the case of unirradiated sensors the fluence $\phi$ is zero, so the trapping time $\tau$ and the 
collecting distance $d$ are infinite. Working out the limits the $CCE$ is found to be:
\begin{equation}
\begin{split}
\lim_{\phi \to 0}CCE & \sim \\ \lim_{d_e,d_h\to \infty}   \Big\{\Big[\dfrac{d_e+d_h}{w}\Big]&-\Big(\dfrac{d_e}{w}\Big)^2\Big(\dfrac{w}{d_e}-\dfrac{1}{2}\big(\dfrac{w}{d_e}\big)^2+\dfrac{1}{6}\big(\dfrac{w}{d_e}\big)^3\Big)-\Big(\dfrac{d_h}{w}\Big)^2\Big(\dfrac{w}{d_h}-\dfrac{1}{2}\big(\dfrac{w}{d_h}\big)^2+\dfrac{1}{6}\big(\dfrac{w}{d_h}\big)^3\Big)\Big\}=\\
&\dfrac{1}{2}+\dfrac{1}{2}1-\dfrac{w}{6}\Big(\dfrac{1}{d_e}+\dfrac{1}{d_h}\Big)=1-\dfrac{w}{6}\Big(\dfrac{1}{d_e}+\dfrac{1}{d_h}\Big)
\end{split}
\label{eq:CCE_unirr}
\end{equation}
To summarise: $CCE(\phi\to 0)\sim1-\dfrac{w}{6}\Big(\dfrac{1}{d_e}+\dfrac{1}{d_h}\Big)$; for infinite 
collecting distances the $CCE$ is exactly one. 
It is interesting to notice that there's a contribution 1/2 from electrons and 1/2 from holes to the 
$CCE$. Neglecting holes in case of no  trapping for electrons the $CCE$  would be merely 1/2: 
all the holes would be trapped and screen on average half of the total charge $Q_0$.

\subsubsection{Large fluences}
In the case of sensors irradiated to large fluences  the 
collecting distance $d$ is negligible with respect to the sensor thickness $w$.

Working out the limits the $CCE$ is found to be:
\begin{equation}
\begin{split}
\lim_{\phi \to \infty}CCE & \sim \\ 
\lim_{d_e,d_h\to 0}  \Big\{\Big[\dfrac{d_e+d_h}{w}\Big]& - \Big(\dfrac{d_e}{w}\Big)^2 -\Big(\dfrac{d_h}{w}\Big)^2\Big\}\\
\sim & \Big[\dfrac{d_e+d_h}{w}\Big]
\end{split}
\label{eq:CCE_irr}
\end{equation}

To summarise: $CCE(\phi\to \infty)\sim\Big[\dfrac{d_e+d_h}{w}\Big]$; for zero 
collecting distances the $CCE$ is exactly zero. 
The result  in Equation~\ref{eq:CCE_irr} means that in case of heavily irradiation the charge 
can be effectively collected only within a distance from the electrode that is equal to the collection distance $d$; all the remaining $w-d$ have a negligible contribution. As for the unirradiated case, if holes are neglected the $CCE$ would be only about 1/2 of what it should be (in general
 the collection distances are different for electrons and holes).

\subsection{Estimates for some scenarios}

In Table~\ref{tab:cce_scenarios} some estimates for the CCE at certain fluences $\Phi$ for 
certain diode thicknesses $w$ are reported.

\begin{table}[!htbp]
\centering
\begin{tabular}{ccc}
\hline
$w$ [$\mu$m] & $\Phi$ [1$\times10^{15}$~n$_{\rm eq}$/cm$^2$] & CCE [\%] \\
\hline
\hline
50 & 3 & 75 \\
50 & 5 & 	63 \\
50 & 7 & 54 \\
50 & 	10 & 	45 \\
50 & 15 & 34 \\
50 & 20 & 27\\
\hline
100 & 1 & 82 \\
100 & 3 & 59 \\
100 & 5 & 	45 \\
100 & 7 & 36 \\
100 & 10 & 27 \\
100 & 20 & 15 \\
\hline
200 & 0.1 & 96\\
200 & 0.2 & 92\\
200 & 0.5 & 82\\
200 & 1 & 69 \\
200 & 3 & 40 \\
200 & 5 & 	27 \\
200 & 7 & 21 \\
200 & 10 & 15 \\
\hline
\end{tabular}
\caption{\label{tab:cce_scenarios}CCE estimates at fluences $\Phi$ for 
certain diode thicknesses $w$.}
\end{table}

It can be seen that already for moderate fluences, 1-2$\times10^{14}$~n$_{\rm eq}$/cm$^2$, 
the signal amplitude loss is sizeable if the detector is 200~$\mu$m thick. 
For fluences in the range of 5-10$\times10^{15}$~n$_{\rm eq}$/cm$^2$ only very thin diodes 
can still deliver more than 1/4 of the original signal amplitude. 

It is interesting to observe that for 1-2$\times10^{16}$~n$_{\rm eq}$/cm$^2$ there's little or no 
difference in expected CCE between 100 and 50 $\mu$m thick diodes, since charge is collected 
within a very thin layer close to the collected electron, the thickness of this layer being smaller 
than the diode one $w$.

\chapter{Trap occupation probability}
\label{sec:trapoccprob}

\chapter{Additional Model Predictions}
\label{sec:additional}

%\chapter{Flavour Tagging Algorithm}
	%\blindtext
	%On rappelle que \gls{alpha} et \gls{gamma} sont liés par la relation~\eqref{eq:alphagamma}. Pour plus de détails, voir page~\pageref{eq:alphagamma}.


	%\blindtext
