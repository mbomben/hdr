\chapter*{Foreword}
\addcontentsline{toc}{chapter}{Foreword}

I have started my career in experimental particle physics in November 2002, 2 days after having passed 
my last exam as  undergraduate student. I joined the \babar\ group of the ``Universit\`a degli Studi 
di Trieste'' to work on the semileptonic decay $\Bzb\to D^{*+}\ell^{-}\bar{\nu}_{\ell}$. That analysis 
was part of my ``tesi di laurea'' (master degree thesis); I studied the effects of hadronic form factor 
uncertainties on the decay branching ratio
 and the module of the Cabibbo Kobayashi Maskawa element $V_{cb}$. 
 Everything was new and exciting for me; I remember 
with particular pleasure participating in collaboration meetings at SLAC, listening to interesting 
discussions about Monte Carlo, muon efficiencies (quite problematic...), soft pions, shape variables, 
trees and penguins... and then staying up late at night to understand the results of data fits with PAW 
and Minuit.
\newline The work was done under the supervision of Prof. Livio Lanceri, who was Physics Analysis Coordinator 
of the \babar\ collaboration at that time;   I defended my ``tesi di laurea'' in November 2003, 
one week before the exam for the admission to the graduate program in physics. I got admitted 
and decide to keep working with the Trieste \babar\ group.
\\ \\
During the first year of the PhD, 2004, I have worked on the completion of the analysis which was 
then published~\cite{PhysRevD.71.051502}; at that time it was the most precise determination of the most 
abundant neutral $B$ meson decay - quite a nice start! 
Since the uncertainties due to the decay form factors were among the largest  it was decided 
to determine at the same time the form factors and  the branching ratio of the 
$\Bzb\to D^{*+}\ell^{-}\bar{\nu}_{\ell}$, together with $|V_{cb}|$~\cite{PhysRevD.77.032002}. 
That work was carried together with the \babar\ Trieste colleagues Dott. Fabio Cossutti and Dott. 
Giuseppe Della Ricca.
\\During the second year of my PhD I had the opportunity to spend 5 months at SLAC, as 
the on-call operation manager and data quality responsible of the \babar\ silicon vertex tracker (SVT). 
I was very intimidated by the task since the SVT was {\it the} \babar\ subdetector: its five shiny layers 
of double sided silicon microstrip sensors were making possible fundamental studies like \CP violation in $\bpsiks$. It was a though period since the data taking was restarting after a stop of one year and there 
was a lot of pressure to take a lot of data. The PEP-II accelerator was breaking luminosity records 
one after the other but at the cost of problematic data taking conditions (high background, doses 
and dead time) for the SVT. Nonetheless I have enjoyed enormously that period; understanding 
the detector behaviour, working in close contact with colleagues of all around the world and discussing 
about possible improvements for the detector was the reason for keep working hard and doing my best 
despite the stressful situation. I remember in particular trying to understand the increase of the modules 
leakage current of one the external layers~\cite{babar_leakage} made me interested about silicon detectors.
\\
\\The topic of my PhD thesis was the measurement of \CP violation in colour suppressed $b\to c$ decays.
Working together with Dr. Chih-hsiang Cheng and Dr.  Vitaly Eyges I had the opportunity to publish 
the first analysis of the \BDh\ channel at the flavour factories~\cite{PhysRevLett.99.081801}.
\\ I successfully defended my PhD thesis in April 2007, again under the supervision of Prof. Livio Lanceri;
  by that moment on I have focused only on 
silicon tracking detector development for high luminosity colliders, which is indeed the subject of this 
manuscript, prepared to obtain the  ``Habilitation \`a Diriger des Recherches''.
\\
\\
At that time I had decided to change research topic because I was interested in working on an experiment 
{\it from the beginning}, so to say. Part of the   \babar\ Trieste group was already involved in an R\&D 
effort about low mass silicon tracking system, the SLIM5 project. 
Together with Prof. Luciano Bosisio, Dott. Lorenzo Vitale, Dott. Irina Rashevskaya and Dott. Gabriele 
Giacomini I worked on characterising novel thin silicon strip detectors. Other than laboratory activity 
I have contributed to develop the reconstruction, analysis and simulation software for the SLIM5 
demonstrator 2008 beam test~\cite{BETTARINI2010942}, together with the colleagues from the Pisa group 
Dr. Nicola Neri and Dr. John Walsh.  
The experience within the SLIM5 project  was great since I had the opportunity to follow almost all the  
project aspects concerning the strip detectors. 
\\
\\
The SLIM5 project was also the opportunity to work with Dr. Giovanni Marchiori, which I  joined in 
September 2010 here 
at the ``Laboratoire de Physique Nucl\'eaire et de Hautes Energies'' (LPNHE), to work on LHC
radiation hard silicon pixel detectors for the High Luminosity phase of the  ATLAS detector at the CERN  
Large Hadron Collider (LHC). Under the direction of my former \babar\ colleague and \babar\ SVT coordinator
Dr. Giovanni Calderini I started investigating silicon detectors and test structures in the LPNHE clean room 
  and working on TCAD (Technology Computer Aided Design) device simulations.
We were working on the development of $n-on-p$ planar pixels within the ATLAS Upgrade Planar Pixel 
Sensor (PPS) R\&D Project and in 2011 I was asked to become the beam test coordinator of the PPS 
collaboration. Over the years I had the opportunity to work with many master and PhD students from
 European, American and Japanese research Institutes, in CERN North Area experimental areas and at the DESY 
 beam test facility. 
I was the coauthor of the publication summarising  the PPS group beam test 
results. Together with Dr. Jens Weingarten of the  II. Physikalischen Institut, G\"ottingen University 
I have coordinated the work of two PhD students and published the results in 2012~\cite{1748-0221-7-10-P10028}. \\
Many more students defended their master and PhD thesis which were based on data collected at 
beam tests coordinated by me.
\\ \\
At the LPNHE I have supervised  6 internships of undergraduate students since 2011; 
they worked with me on different topics, like measuring silicon detectors
 in the LPNHE clean room, performing TCAD 
simulation of edgeless pixel sensors\footnote{sensor characterised by a very slim un-instrumented area at 
the detector periphery}~\cite{bib:nim2012} for the 
High Luminosity LHC (HL-LHC), improving the clean room equipment  and characterising pixel sensors 
prototypes. Other students from the LPNHE laboratory worked with me, either
in clean room or at beam tests, for internships and PhD.
\\
Among those students Audrey Ducourthial decided to continue working with me, and since October 2015 she is 
 PhD student under my supervision in the LPNHE ATLAS group. 
Her main research topic is the development of silicon pixel sensors for the future ATLAS tracker, intended 
for the HL-LHC phase, and the analysis of the $H$ Higgs boson decay to $b$ quarks
 $H\to\b\bbar$ channel.
\\She participated in many beam tests, measuring 
edgeless pixel prototypes developed in collaboration with the FBK foundry, and later reconstructing 
and analysing the data; the results have been recently published~\cite{1748-0221-12-05-P05006}. 
She also collaborates to the modelization of radiation damage to pixel detectors: since 2017 
an ATLAS Pixel sub-working group has been formed whose goal is to include in ATLAS Monte Carlo 
simulations the effects due to radiation damage to the ATLAS Inner Detector. Together 
with Dr. Benjamin Nachman of Lawrence Berkeley National Laboratory I coordinate this 
ATLAS working group  which more than 10 master and PhD students, Audrey Ducourthial included, collaborate to.
\\
Radiation damage effects to pixel detector could lead to degradation of vertexing, tracking 
and jet flavour tagging performance. Together with Audrey  Ducourthial I will   work on assessing the impact of this 
degradation, in particular in the channel $H\to\b\bbar$. 
\\Audrey  Ducourthial also started working on the optimisation of the jet flavour tagging algorithms for the
HL-LHC  ATLAS detector. 
\\
\\
\\
\\
{\it Memoir organisation}
\\
\\This memoir, prepared to obtain the ``Habilitation \`a Diriger des Recherches'', is the summary of my 
research activities after the PhD graduation. After an introduction to the context of my researches 
(Chapter~\ref{chap:context}) a discussion on silicon detectors will follow in Chapter~\ref{chap:silicon}.
The motivations and main results of the SLIM5 project will be presented in Chapter~\ref{chap:SLIM5}, 
with a detailed discussion on strip detectors spatial resolution.
\\TCAD simulations are presented in Chapter~\ref{chap:TCAD}, together with some case studies 
and applications.
Chapter~\ref{chap:ATLAS} summarises  the LHC physics program, the ATLAS detector and 
in particular its pixel detector; it serves as an introduction to the next Chapter (\ref{chap:digi}) where 
the modelization of radiation damage in detectors is presented, together with results for the 
current ATLAS data taking.
\\The new ATLAS Inner Tracker, intended for the Phase-II of the experiment, will be presented in 
Chapter~\ref{chap:ITk}; radiation hard and edgeless pixel sensors will be discussed in detail. 
\\ Finally research perspectives (Chapter~\ref{chap:perspectives}) and a general summary of my past, present 
and planned activities will be given.
\\
\\
Whenever possible I have tried to include background material as an introduction to my research topics. 
Original contributions are to be found in Chapter~\ref{chap:SLIM5} (Sections \ref{sec:SLIM5Results}-\ref{sec:Slim5Summary}), Chapter~\ref{chap:TCAD} (Sections~\ref{sec:TCADIntro}-\ref{sec:TCADfund},~\ref{sec:Savic},~\ref{sec:TCADComparison},~\ref{sec:TCADSummary}), Chapter~\ref{chap:digi} and in 
Chapter~\ref{chap:ITk} (Sections~\ref{sec:radhardpixels}-\ref{sec:itksummary}).



