\chapter{Silicon Detectors for High Energy Physics}
\label{chap:silicon}

Pixel and strip detectors realised on high resistivity silicon substrates are nowadays the standard 
choice for high energy physics experiments. 
In this Chapter an introduction to silicon detectors will be given, focusing 
on those aspects that are relevant for the purpose of tracking and vertexing.
Excellent books on the subject exists, like~\cite{Lutz:411172,Sze1981,Wang1989,Krammer}. Here 
some extracts from those will be reported, just to introduce the subject. 
After reviewing the basics  a discussion on radiation damage in silicon 
will follow in Section~\ref{sec:RadDam}.
\section{Semiconductor Basics}

\subsection{Crystals and Energy Bands}

The physics of semiconductor devices is naturally dependent on the physics of semiconductor 
themselves~\cite{Sze1981}. In this brief introduction only crystalline semiconductors will be treated, 
with a particular focus on silicon. Most commonly used semiconductors are crystals with 
diamond (Si and Ge) or zinc blende ({\it e.g.} GaAs) lattice type. In Figure~\ref{fig:diamondLattice} 
a schematic view of the diamond lattice is presented.


\begin{figure}[htbp]
   \centering
   \includegraphics{diamondLattice.pdf} % requires the graphicx package
   \caption{\label{fig:diamondLattice}Diamond (a) and zinc blend (b) lattice. (After~\cite{Lutz:411172})}
\end{figure}

Due to the Pauli exclusion principle, electrons in crystals are organised in energy bands, 
each one containing many closely spaced levels; Figure~\ref{fig:EnergyLevels} helps in picturing 
the situation for diamond lattice. At very large distances each atom has the same two energy levels; 
the energy levels are $N$-fold degenerate ($N$ being the number of atoms), they indeed split 
into $N$ closely spaced levels when the atoms are brought close together. 
For $N\to\infty$, one speaks of energy bands, rather than levels, and these bands broaden, merge 
and split again with even closer spacing~\cite{Lutz:411172}.

\begin{figure}[htbp]
   \centering
   \includegraphics{EnergyLevels.pdf} % requires the graphicx package
   \caption{\label{fig:EnergyLevels}Energy levels of silicon atoms arranged in a diamond structure, as a function of lattice spacing. (After~\cite{Lutz:411172})}
\end{figure}

The spacing corresponding to silicon is indicated in Figure~\ref{fig:EnergyLevels} and corresponds to 
the minimum total energy of the electrons and the lattice, not very far from the minimum energy of
 the electrons in the filled valence band. 
 At low temperature one has a completely filled valence band and an empty conduction band; at room 
 temperature the thermal energy is high enough to lift a few electrons to the conduction band, thus 
 creating a weak conductivity due to free electrons and electrons vacancies, {\it i.e.} holes.
In Figure~\ref{fig:EnergyBandGap} the energy band structures of several materials are reported. 
From that Figure it can be seen that in semiconductors at low temperature one has a completely filled valence band and an empty conduction band; at room 
temperature the thermal energy is high enough to lift a few electrons to the conduction band, thus 
creating a weak conductivity due to free electrons and holes.
 
 \begin{figure}[htbp]
   \centering
   \includegraphics[width=0.8\textwidth]{EnergyBandGap.png} % requires the graphicx package
   \caption{\label{fig:EnergyBandGap}Energy band structure of several materials. For 
   semiconductors the  $T=0$~K and $T>0$~K situations are reported; for metals two 
   possible band configurations are represented (After~\cite{Krammer}).}
\end{figure}


 The structure of an isolator, or insulator, is similar,  except that the band gap is much larger so that 
 the occupation probability of states in the conduction band is zero.   Conductors may either have 
 overlapping valence and conduction bands  or a partially filled conduction band. 
 We can conclude that the main difference between conductors, semiconductors and insulators 
 is the value of the band gap energy $E_g$.
 
 \begin{figure}[htbp]
   \centering
   \includegraphics[width=0.55\textwidth]{KinEnergy.pdf} % requires the graphicx package
   \caption{\label{fig:KinEnergy}Potential and kinetic energy in the band 
   representation (After~\cite{Lutz:411172}).}
\end{figure}

Focusing on the dynamics of carriers in crystalline materials, it can be proven that  electrons in the 
conduction band and holes in the valence band are similar to free particles  but with an effective mass 
($m_n^*$, $m_p^*$) different from elementary electrons not imbedded in the lattice.
This mass is furthermore dependent on other parameters such as the direction of movement with 
respect to the crystal axis. The kinetic energy of electrons is measured from the lower edge of the 
conduction band upwards, that of the holes downward from the upper edge of the valence band; 
Figure~\ref{fig:KinEnergy} presents the energy diagram for free electrons and holes in lattice.

This simplified picture presents important limitations; in particular it neglects the relative position 
in lattice reciprocal space of the minimum conduction band and the maximum of the valence band. 
If there is no difference among the two positions then the semiconductor is said to have ``direct'' 
bandgap; otherwise it is an indirect semiconductor. Figure~\ref{fig:bandStructures} shows the difference 
between indirect semiconductors, like Silicon and Germanium, and direct ones, like Gallium Arsenide. 


 \begin{figure}[htbp]
   \centering
   \includegraphics[width=0.55\textwidth]{bandStructures.jpg} % requires the graphicx package
   \caption{\label{fig:bandStructures}Germanium (left), Silicon (center) and Gallium Arsenide (right) band structures. (Bottom) valence bands; (top) conductive bands.}
\end{figure}

For indirect semiconductors, the process of creation or annihilation of an electron-hole pair requires 
not only a quantum of energy, like a photon, but also a net lattice momentum transfer, thanks to 
phonons. In Silicon at room temperature the bandgap energy value is of about $E_g\sim1.12$~eV, 
while the mean ionization energy is $\epsilon~3.6$~eV: the difference is due to the distance in the 
lattice reciprocal space of the edge of conductive and valence band. 

\subsection{Extrinsic Semiconductors and Doping}

Intrinsic semiconductors contain a very limited number of impurities  compared with the number of 
thermally generated electrons and holes. 
Electron states with energy $E$ are occupied following the Fermi-Dirac statistics:
\begin{equation}
F(E)=\dfrac{1}{1+\exp{\Bigg(\dfrac{E-E_F}{kT}}\Bigg)}
\label{eq:FermiDirac}	
\end{equation}
where $E_F$, the Fermi energy, is the energy at which the occupation probability of a (possible) 
state is one half, $k$ is the Boltzmann constant and $T$ is the absolute temperature.
In Intrinsic semiconductors electrons and holes exist on account of thermal creation of electron-hole 
pairs, so we have:

\begin{equation}
p=n,
\label{eq:n=p}
\end{equation}

{\it i.e.} the concentration of electrons $n$ equals that of holes $p$. 
We will assert the mass action law for semiconductors:
\begin{equation}
np=n_i^2
\label{eq:massLawAction}
\end{equation} 
where $n_i$ is the {\it intrinsic carrier concentration}. The intrinsic carrier concentration depends only 
on the temperature $T$, the effective mass of the carriers $m^*$ 
and the band gap energy 
$E_g$~\cite{Lutz:411172}.


Intrinsic semiconductors are rarely used in semiconductor 
devices since it is extremely difficult to obtain sufficient purity in the material. Moreover, in most cases 
one intentionally alters the property of the material by adding small fractions of specific impurities. 
This procedure is called doping.  Doping is the replacement of a small number of atoms in the lattice by 
atoms of neighbouring columns from the atomic table (with one valence electron more or less compared 
to the basic material). Depending on the type of added material, one obtains n-type 
semiconductors with an excess of electrons in the conduction band or p-types with additional holes in 
the valence band. 

Doping Silicon with an element of the V group (P, As, Sb) leaves a valence electron of dopant atom 
loosely bound; those atoms are identified  
as {\it donor} dopants. The energy level of the donor is just below the edge of the conduction band; 
at room temperature most electrons are raised from the donor dopant to the conduction band.  
The doping with donors  is illustrated in Figure~\ref{fig:nDoping}. A semiconductor doped with 
donors is called a $n-$type semiconductor. There is an imbalance between 
electrons over holes in $n$-type semiconductors; electrons are the majority carriers, while holes the
minority ones. 

 \begin{figure}[htbp]
   \centering
   \includegraphics[width=0.45\textwidth]{nDopingBonds.pdf} 
   \includegraphics[width=0.35\textwidth]{nDopingBands.pdf} 
   \caption{\label{fig:nDoping}Doping Silicon with donor atoms. (Left) atom bonds with donor dopant; 
   (right) energy bands diagram after donor doping. (After~\cite{Krammer}).}
\end{figure}

Doping Silicon with an element of the III group (B, Al, Ga, In) leaves one valence bond open; 
those atoms are identified  as {\it acceptor} dopants. 
The energy level of the acceptor is just above the edge of the valence band; 
at room temperature most levels are occupied by electrons leaving holes in the valence band.  
The doping with acceptors  is illustrated in Figure~\ref{fig:pDoping}. A semiconductor doped with 
acceptors is called a $p-$type semiconductor. There is an imbalance between 
holes over electrons in $p$-type semiconductors; holes are the majority carriers, while electrons the
minority ones.



 \begin{figure}[htbp]
   \centering
   \includegraphics[width=0.45\textwidth]{pDopingBonds.pdf} 
   \includegraphics[width=0.35\textwidth]{pDopingBands.pdf} 
   \caption{\label{fig:pDoping}Doping Silicon with acceptor atoms. (Left) atom bonds with acceptor 
   dopant; (right) energy bands diagram after acceptor doping.(After~\cite{Krammer}).}
\end{figure}
In a doped semiconductor the relation $n=p$ doesn't hold, while the mass action law  
(Eqution~\ref{eq:massLawAction}) still does. Semiconductors where $n\neq p$ are called
 {\it extrinsic}.
In a doped semiconductor electrons are merely redistributed among the various energy states, 
but not taken out of or put into the semiconductor itself, the crystal remains electrically neutral. 
The equation that states this charge-neutrality condition reads:

\begin{equation}
n+N_a^-=p+N_d^+,
\label{eq:chargeNeutrality}
\end{equation}

where $N_{a(d)}^{-(+)}$ represent the charge density of ionised acceptors (donors) respectively. 
At room temperature dopants are normally ionised so it is safe to assume that $N_a^-\simeq N_a$ 
and $N_d^+\simeq N_d$, hence $n-p=N_d-N_a$. From charge neutrality and mass action law  
it can be easily shown that for an $n$-type semiconductor the concentration of electrons $n$ is 
equal to that of the donor dopants $N_d$ to a very good level; with the same reasoning in a 
$p$-type semiconductor the concentration of holes $p$ is equal to that of the acceptor 
dopants $N_a$. 
Before moving to the case of semic


\subsection{Carrier Transport in Semiconductors}

\section{The p-n Junction}

At the interface of an $n$-type and $p$-type semiconductor the difference in the Fermi levels cause 
diffusion 
of surplus carries to the other material until thermal equilibrium is reached. At this point the Fermi level 
is equal. The remaining ions create a 
space charge and an electric field stopping further diffusion. 

\begin{figure}[htbp]
   \centering
   \includegraphics[width=0.45\textwidth]{p_close_to_n.pdf} 
   \includegraphics[width=0.45\textwidth]{pn_junction.pdf} 
   \caption{\label{fig:pnJunction}P-n junction formation. (Left) Two oppositely doped semiconductors 
   are compared. (Rigth) The p-n junction is formed. (After~\cite{Krammer}).}
\end{figure}

\section{Why Use Silicon}
Let's know focus only on Silicon. Silicon detectors replaced the  gas based detectors in the tracking systems, since they offer a much
 better position information and an improved energy resolution. The reasons for this are to be found 
in the large density of silicon at room temperature, in the relatively low mean ionisation energy and 
in the possibility of use photolithography to realise charge collecting electrodes. 
These three characteristics allow to have large signals with a small active thickness and 
excellent spatial resolution. Some of the Silicon properties that are relevant for high energy 
physics applications are summarised in Table~\ref{tab:SiProperties}.

% Requires the booktabs if the memoir class is not being used
\begin{table}[htbp]
   \centering
   %\topcaption{Table captions are better up top} % requires the topcapt package
   \begin{tabular}{@{} lcr @{}} % Column formatting, @{} suppresses leading/trailing space
      \toprule
      \multicolumn{3}{c}{Silicon} \\
      \cmidrule(r){1-3} % Partial rule. (r) trims the line a little bit on the right; (l) & (lr) also possible
      Feature    & Value & Comments \\
      \midrule
      Density  $\rho$    & 2.33~g/cm${^3}$ & compact and thin detectors  \\
      Energy bandgap $E_g$ & 1.12~eV & non-cryogenic operation \\
      Mean ionisation energy $\epsilon$ & 3.6~eV & large signals\\
      Radiation length $X_0$      &  9.37~cm & thin detectors to minimize  \\
                                       &                 & multiple scattering \\
      Electron mobility  $\mu_e$     & $\sim$1350 cm$^2$V/s  & fast charge collection \\
      Saturation velocity $v_{sat}$ & $\sim$10$^{7}$ cm/s & fast charge collection \\
      \bottomrule
   \end{tabular}
   \caption{Summary of silicon properties relevant for high energy physics applications~\cite{Lutz:411172}.}
   \label{tab:SiProperties}
\end{table}

Other important characteristics that can explain the success of silicon are its large abundance, 
the possibility of changing its properties by doping, the existence of a natural oxide~\cite{Hartmann2012}.

\section{Silicon Trackers}
\section{Radiation Damage}
\label{sec:RadDam}
