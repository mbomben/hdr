\chapter{Silicon Detectors for High Energy Physics}
\label{chap:silicon}

Pixel and strip detectors realised on high resistivity silicon substrates are nowadays the standard 
choice for high energy physics experiments. 
In this Chapter an introduction to silicon detectors will be given, focusing 
on those aspects that are relevant for the purpose of tracking and vertexing. After reviewing the basics of the subject a discussion on radiation damage in silicon 
will follow in Section~\ref{sec:RadDam}.
\section{Semiconductor Basics}

\section{Why Use Silicon}

Silicon detectors replaced the  gas based detectors in the tracking systems, since they offer a much
 better position information and an improved energy resolution. The reasons for this are to be found 
in the higher density of silicon at room temperature and to the possibility of use photolithography 
to realise charge collecting electrodes. 


\section{The p-n Junction}
\section{Silicon Trackers}
\section{Radiation Damage}
\label{sec:RadDam}
