\chapter{SLIM5}
\label{chap:SLIM5}
In this Chapter the SLIM5 R\&D project will be presented. After a short summary of the project 
motivations, goals and timeline (Section~\ref{sec:SLIM5Project}), the Physics motivations of 
and the tracker requirements for SuperB and ILC will be briefly discussed. 
The pixel  (Section~\ref{sec:Apsel4D}) and strip (Section~\ref{sec:Striplets}) 
detector prototypes developed 
within the SLIM5 will be presented. Finally, in Section~\ref{sec:SLIM5Results} the results 
in~\cite{BETTARINI2010942,BOMBEN2010159} will be discussed.


\section{The SLIM5 Project}
\label{sec:SLIM5Project}
The SLIM5 project~\cite{SLIM5:proj} aimed at advancing the state-of-the-art in the development of thin 
tracking systems to be applied in High-Energy Physics. The project was financed for three years
 (2006-2008) by INFN - National Scientific Committee 5~\cite{INFN_V} and involved several 
 italian research institutes. 
 
 The SLIM5 collaboration worked on developing tracking systems for experiments at 
 high luminosity flavour 
 factories, like the proposed SuperB (\cite{Baszczyk:2013xua}) and linear colliders. The goal was to deliver thin silicon tracking detectors with possibility 
 of self-triggering thanks to the combination of data-driven data acquisition and pattern matching 
 algorithm with very low latency.
Let's know see the required performance for such detectors. 

\section{Tracking and Vertexing Requirements for SuperB and ILC Experiments}
Experiments at high luminosity colliders have to accomplish a high precision measurements 
exploiting at maximum the large dataset they are expected to integrate. 
This means that the experiments have to be very efficient and show excellent performance, 
even in presence of a very intense particle rate; this is particularly true for the tracking and 
vertexing detectors, which are the closest to the beams interaction point. It has also to be stressed 
that with sub-optimal tracking and vertexing performance there's no physics case for such 
experiments; hence the tracking and vertexing detectors are the crucial parts of experiments at 
high luminosity colliders.

We now review quickly some physics cases for SuperB and ILC and the related constraints on 
tracking detectors.
\subsection{SuperB}

\subsection{ILC}
\cite{ILCVertexing2007}


\section{The CMOS MAPS Apsel4D}
\label{sec:Apsel4D}

\section{The Striplets Detector}
\label{sec:Striplets}

\section{Discussion of the Performance of SLIM5 Detectors}
\label{sec:SLIM5Results}
A demonstrator was built~\cite{BETTARINI2010942}, it was composed of CMOS MAPS 
 and thin silicon DSSDs and tested it on beam at the T9 facility of the CERN PS. 


\section{Summary}
