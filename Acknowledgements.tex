%\begin{titlepage}
\vspace*{4cm}
\begin{flushright}
{\itshape A Simonetta
}
\end{flushright}
%\end{titlepage}



\chapter*{Acknowledgements}

I want to start by acknowledging the {\it rapporteurs}, Fran\c{c}ois Le Diberder, Achille Stocchi and Marc 
Winter, for accepting 
to be referees of this  report, prepared to obtain the  ``Habilitation \`a Diriger des Recherches''; I 
also express my gratitude to Daniela Bortoletto, Dominik Dannheim  and Anna Macchiolo, who 
accepted to be members of the {\it jury}.

I want to thank the researchers I worked with in the Trieste \babar\ and SLIM5 group, 
in particular Livio Lanceri and Lorenzo Vitale, for mentoring me and being excellent colleagues first 
and later good friends. 

\noindent Luciano Bosisio deserves a special mention: his profound knowledge of silicon detectors 
 motivated me to work on this topic; I want to thank him for all he taught me, in Trieste first and later 
 in Paris.
I also want  to express my gratitude to Professor Paolo Poropat: my first steps in
physics career are due to his dedication to physics and passion for teaching; you left us too soon.

In the ATLAS group of ``Laboratoire de Physique Nucl\'eaire et de Hautes Energies'' (LPNHE) I have
the chance to work with excellent researchers, in particular Giovanni Calderini and Giovanni Marchiori. 
Giovanni Calderini mentored and supported me in the last seven years, giving me the opportunity  
to deepen my knowledge of silicon detectors and to progress in my career. 
Working with Giovanni Marchiori is  always an enriching experience: I am indebted to him for 
many things I learnt, and for the good times too. 
I want to thank Francesco Crescioli too; we worked together at the time of SLIM5 and now in 
ATLAS R\&D projects; this was and is great. 

My gratitude goes to all the groups I collaborated to within the \babar, ATLAS and RD50 collaboration. 
In particular I want to thank Michael Moll (RD50) for his support and Benjamin Nachman (ATLAS); 
they kindly reviewed part of this report and provided valuable feedback which helped improving its quality. 
I want to thank also Jens Weingarten and Andr\'e Rummler; we worked hard but always in a nice 
atmosphere.
 

Without the encouragement of my family none of the achievements I got were possible. 
My family followed and supported 
me all along my career, as a student first and then as a researcher. I have learnt from them to work hard 
and to pursue excellence. So {\it grazie} Luciano, Daniela, Alberto and Raffaella, and {\it ciao} Luigi, Celeste 
and Angela. My gratitude goes also to all my relatives, too numerous to be listed here.

Friends are of course fundamental and I feel very luck in having found a very good one: Andrea; thank you 
for being such an unbelievable friend for all these years. I do feel very lucky. 


Simonetta, the support and help I got from you cannot be quantified (and I am physicist!).  You encouraged 
me enormously at the beginning of this {\it parisien} adventure,  reassuring me that I was making the right 
choices, and you kept supporting me whenever I felt in difficulty. Thank you for all your love and  I close here 
the list of acknowledgements, simply saying that I love you.











